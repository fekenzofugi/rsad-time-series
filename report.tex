\documentclass[12pt]{article}
\setlength{\parskip}{0.5cm}
\usepackage[margin=1in]{geometry}
\usepackage{setspace}
\usepackage{titling}
\usepackage{lmodern}
\usepackage{ragged2e}
\usepackage{graphicx}
\usepackage{amsmath}
\usepackage{amssymb}
\usepackage{float}

\begin{document}

\begin{titlepage}
    \centering
    \begin{figure}
        \centering
        \begin{minipage}{0.4\textwidth}
            \centering
            \includegraphics[width=0.8\textwidth]{figures/logo_nipe.png}
        \end{minipage}
        \hfill
        \begin{minipage}{0.3\textwidth}
            \centering
            \includegraphics[width=0.8\textwidth]{figures/unicamp.png}
        \end{minipage}
    \end{figure}

    {\Large \textbf{Research Internship Abroad (BEPE/FAPESP)}}\\[1cm]

    {\LARGE \textbf{Remote Sensing Anomaly Detection in Time-Series Imagery and Application Development in the Agricultural Context}}\\[1.5cm]

    {\large Process: 2025/22483-0}\\[0.5cm]

    {\large Final Report: December/01/2025 to March/01/2026 (3 months)}\\[3cm]

    \begin{flushright}
    {\large \textbf{Brazil Advisor:} Dr. Rubens Augusto Camargo Lamparelli}\\
    {\large \textbf{UK Supervisor:} Dr. Jefersson Alex dos Santos}\\
    {\large \textbf{UK Co-Supervisor:} Prof. Po Yang}\\
    {\large \textbf{Fellow:} Fernando Kenzo Imami Fugihara}\\
    \end{flushright}

    \vfill

    {\large{Campinas, SP}}\\
    {\large{2025}}
\end{titlepage}

\pagenumbering{gobble}
\newpage
\pagenumbering{arabic}

\section*{Abstract}

This report documents the theoretical foundations, methodological approaches, and 
findings from the the internship (December 1 - March 1), focusing on time-series
analysis fundamentals, anomaly detection algorithms.

\clearpage

\tableofcontents

\clearpage

\section{Introduction}

\subsection{Research Context and Objectives}
This research internship focuses on the detection of anomalies in agricultural
systems using time-series analysis of Sentinel-2 satellite imagery. The primary
objectives are:

\begin{itemize}
    \item \textbf{Proof of Concept:} Verify if its possible to detect pest-net installations in plasticulture 
    systems in Mossoró, Brazil, using time-series analysis of Sentinel-2 imagery.
    \item \textbf{Proof of Concept:} Verify if its possible to detect pest infestations, such as Fall Armyworm
    (Spodoptera frugiperda) in maize crops, using time-series analysis of Sentinel-2 imagery.
    \item \textbf{Algorithm Development:} Model anomaly detection algorithms capable of 
    identifying both pest-net installations and pest infestation events in time-series 
    satellite imagery.
    \end{itemize}

In the context of learning anomaly detection techniques applied to remote sensing, the research 
objectives also align with the Brazil–UK–Africa collaboration “SmartPest-Ghana: Exploring 
LLM-driven mobile solutions for climate-smart pest management in maize farming”, 
funded by UK Research and Innovation (UKRI).

In terms of application development, the internship aims to contribute to the
development of a mobile application that leverages the anomaly detection algorithms
to provide real-time pest management recommendations to smallholder farmers in Ghana.

\begin{itemize}
    \item \textbf{Application Development:} Contribute to the development of a mobile application
     that integrates the Computer Vision and LLM models to provide real-time pest management 
     recommendations to smallholder farmers.
\end{itemize}

\clearpage

\section{Time-Series Analysis Fundamentals}

\subsection{Time-Series Definitions}

\subsubsection{Univariate Time-Series (UTS)}
A univariate time-series represents a single variable (such as NDVI) measured at a single pixel location over time. For a pixel observed at $t$ timestamps, the UTS $X$ is represented as:

\[
X = (x_1, x_2, \ldots, x_t)
\]

where $x_i$ represents the feature value at timestamp $i \in T$ and $T = \{1, 2, \ldots, t\}$.

\subsubsection{Multivariate Time-Series (MTS)}
A multivariate time-series represents multiple variables that exhibit both temporal dependencies (correlations across time) and spatial dependencies (correlations between variables). For a pixel with $d$ features observed over $t$ timestamps:

\begin{equation}
X = (X_1, X_2, \ldots, X_t)
  = \bigl(
    (x_1^1, x_1^2, \ldots, x_1^d),
    (x_2^1, x_2^2, \ldots, x_2^d),
    \ldots,
    (x_t^1, x_t^2, \ldots, x_t^d)
    \bigr)
\end{equation}

where $X_i = (x_i^1, x_i^2, \ldots, x_i^d)$ represents the feature vector at time $i$, and $x_i^j$ is the observation at time $i$ for the $j$-th dimension.

\subsection{Time-Series Components}

Understanding the structural components of time-series data is essential for selecting appropriate analysis and smoothing techniques.

\subsubsection{Level}
The level represents the average value of the time-series and can be conceptualized as the mean around which observations fluctuate.

\subsubsection{Stationarity}
A time-series is stationary when its statistical properties (mean, variance, covariance) remain constant over time. Stationarity is an important assumption for many time-series analysis techniques.

\subsubsection{Trend}
The trend captures long-term systematic increases or decreases in the series. Presence of trend typically violates stationarity, as the mean changes over time.

\subsubsection{Seasonality}
Seasonality refers to regular, predictable patterns that repeat at fixed intervals (e.g., daily, weekly, seasonal cycles in crop development).

\subsubsection{Cyclicity}
Cycles are repetitive patterns similar to seasonality but occurring over longer, less predictable time periods not aligned with calendar intervals. In agricultural contexts, cycles may relate to multi-year climate patterns or economic factors.

\subsection{Time-Series Smoothing Techniques}

Smoothing techniques are essential preprocessing steps that remove noise from time-series data while preserving important patterns. This is particularly critical when comparing pixel time-series or applying distance-based algorithms.

Consider two pixels with similar underlying patterns but different noise characteristics. Without smoothing, distance-based clustering algorithms may incorrectly classify these pixels as dissimilar due to noise rather than genuine differences in behavior. Smoothing addresses this issue by enhancing signal-to-noise ratio.

The choice of smoothing method depends on the structural characteristics of the time-series, as summarized in Table \ref{tab:t1}.

\begin{table}[h]
\centering
\caption{Smoothing algorithm applicability for different time-series structures}
\label{tab:t1}
\renewcommand{\arraystretch}{1.3}
\begin{tabular}{|l|c|c|c|c|}
\hline
\textbf{Algorithm} & \textbf{Level} & \textbf{Trend} & \textbf{Seasonality} & \textbf{Parameters} \\
\hline
Single HWES  & Yes & No  & No  & $\alpha$ \\
Double HWES  & Yes & Yes & No  & $\alpha, \beta$ \\
Triple HWES  & Yes & Yes & Yes & $\alpha, \beta, \gamma$ \\
\hline
\end{tabular}
\end{table}

Table \ref{tab:t2} defines the variables used in smoothing formulations.

\begin{table}[h]
\centering
\caption{Variables utilized in smoothing models}
\label{tab:t2}
\renewcommand{\arraystretch}{1.4}
\begin{tabular}{|c|l|}
\hline
\textbf{Symbol} & \textbf{Description} \\
\hline
$X$ & Observation \\
$S$ & Smoothed observation \\
$B$ & Trend factor \\
$C$ & Seasonal index \\
$F$ & Forecast at $m$ periods ahead \\
$\alpha$ & Data smoothing factor, $\alpha \in (0,1)$ \\
$\beta$ & Trend smoothing factor, $\beta \in (0,1)$ \\
$\gamma$ & Seasonal change smoothing factor, $\gamma \in (0,1)$ \\
$\phi$ & Damped smoothing factor, $\phi \in (0,1)$ \\
$t$ & Time period index \\
\hline
\end{tabular}
\end{table}

\subsubsection{Moving Average and Weighted Average}
These methods compute the future value as the average (or weighted average) of $k$ previous values. While useful for trend observation and feature engineering, these approaches are unsuitable for satellite time-series with trend and seasonality components.

\subsubsection{Single Exponential Smoothing (SES)}
SES models the level component only and is appropriate for stationary series without trend or seasonality. It weights recent observations more heavily based on the assumption that the future is more closely related to the recent past.

\[
S_0 = X_0
\]

\[
S_t = \alpha X_t + (1 - \alpha) S_{t-1}, \quad t > 0, \; 0 < \alpha < 1
\]

\textbf{Application:} Limited applicability to agricultural time-series, which typically exhibit both trend and seasonality.

\subsubsection{Double Exponential Smoothing (DES)}
DES extends SES by incorporating trend, making it suitable for series with trend but without seasonality.

\[
\begin{aligned}
S_0 &= X_0 \\
B_0 &= X_1 - X_0 \\
S_t &= \alpha X_t + (1 - \alpha)(S_{t-1} + B_{t-1}) \\
B_t &= \beta (S_t - S_{t-1}) + (1 - \beta) B_{t-1}, \quad \alpha, \beta \in (0,1)
\end{aligned}
\]

\textbf{Application:} Applicable to agricultural time-series when seasonality is not a dominant factor.

\subsubsection{Triple Exponential Smoothing (TES) / Holt-Winters}
TES represents the most comprehensive smoothing approach, modeling level, trend, and seasonality simultaneously. This method is most appropriate for agricultural satellite time-series, which exhibit all three components.

\[
\begin{aligned}
S_0, F_0 &= X_0 \\
B_0 &= \frac{\sum_{i=0}^{L-1} \left( X_{L+i} - X_i \right)}{L^2} \\
S_t &= \alpha \left( X_t - C_{t \bmod L} \right) + (1 - \alpha)\left( S_{t-1} + \phi B_{t-1} \right) \\
B_t &= \beta \left( S_t - S_{t-1} \right) + (1 - \beta)\phi B_{t-1} \\
C_{t \bmod L} &= \gamma \left( X_t - S_t \right) + (1 - \gamma) C_{t \bmod L} \\
F_{t+m} &= S_t + B_t \sum_{i=1}^{m} \phi^{i} + C_{t \bmod L}, \quad \alpha, \beta, \gamma \in (0,1)
\end{aligned}
\]

\textbf{Application:} Recommended for Sentinel-2 vegetation index time-series, which exhibit clear seasonal patterns related to crop phenology.

\section{Methodology}

\subsection{Data and Study Regions}
This research utilizes Harmonized Sentinel-2 (S2) Level-2A surface reflectance imagery, accessed through Google Earth Engine (GEE). The Copernicus Sentinel-2 mission consists of two polar-orbiting satellites (2A and 2B) operating in sun-synchronous orbit at 786 km altitude, phased 180° apart.

\subsubsection{Mossoró, Rio Grande do Norte, Brazil}

Figure \ref{fig:study_area_mossoro} illustrates the landscape characteristics of this region.

\begin{figure}[H]
    \centering
    \includegraphics[width=0.6\textwidth]{figures/study_area_mossoro.png}
    \caption{Mossoró study region, Rio Grande do Norte, Brazil, showing plasticulture fields.}
    \label{fig:study_area_mossoro}
\end{figure}

\subsubsection{Ejura, Ghana, West Africa}

Figure \ref{fig:study_area_ejura} presents the diverse agricultural landscape of Ejura.

\begin{figure}[H]
    \centering
    \includegraphics[width=0.6\textwidth]{figures/study_area_ejura.png}
    \caption{Ejura study region, Ghana, West Africa, showing diverse crop systems.}
    \label{fig:study_area_ejura}
\end{figure}

\subsubsection{Kurnool and Gadwal Districts, Southern India}

Figure \ref{fig:study_area_india} shows the agricultural landscape characteristics of these districts.

\begin{figure}[H]
    \centering
    \includegraphics[width=0.6\textwidth]{figures/study_area_india.png}
    \caption{Kurnool and Gadwal districts, Southern India, showing maize cultivation areas.}
    \label{fig:study_area_india}
\end{figure}

\subsection{Preprocessing Sentinel-2 Imagery}
For each study region, Sentinel-2 images were filtered by date range and 
cloud cover percentage.

\subsection{Spectral Indices}

Spectral vegetation indices (SVI) derived from remote sensing data provide quantitative 
measures of crop health and stress. These indices are particularly effective for detecting 
impacts of diseases and pest invasions. This research employs multiple indices to cap                               ture 
different aspects of plant health:

\subsubsection{Plastic Indices}
Table \ref{tab:plastic_indices} summarizes the plastic indices used to detect plasticulture systems

\begin{table}[ht]
\centering
\caption{Spectral plastic indices and corresponding equations using Sentinel-2 bands.}
\label{tab:plastic_indices}
\renewcommand{\arraystretch}{1.4}
\small
\begin{tabular}{p{5cm} p{5cm} p{5cm}}
\textbf{Index} & \textbf{Equation} & \textbf{Reference} \\
\hline
Plastic Index & 
$\frac{SWIR - NIR}{SWIR + NIR}$ & 
Mudereri et al., 2022
\end{tabular}
\end{table}

\subsubsection{Vegetation Health Indices}
Table \ref{tab:veg_indices} presents the spectral indices used in this study, along with their mathematical formulations and corresponding Sentinel-2 band combinations.

\begin{table}[ht]
\centering
\caption{Spectral vegetation indices and corresponding equations using Sentinel-2 bands.}
\label{tab:veg_indices}
\renewcommand{\arraystretch}{1.4}
\small
\begin{tabular}{p{5cm} p{5cm} p{5cm}}
\textbf{Index} & \textbf{Equation} & \textbf{Reference} \\
\hline
Normalized Difference Vegetation Index (NDVI) & 
$\frac{NIR - Red}{NIR + Red}$ & 
Rouse et al., 1974 \\

Enhanced Vegetation Index (EVI) & 
$2.5 \frac{NIR - R}{NIR + 6R - 7.5B + 1}$ & 
Huete et al., 1997 \\

Leaf Area Index (LAI) &
$3.618 \times \text{NDVI} - 0.118$ & 
Weiss and Baret, 2016 \\
\end{tabular}
\end{table}

The pest infestation detection model identifies anomalies in vegetation time-series by 
detecting significant NDVI decreases that may indicate pest attacks. The model computes 
the rate of change in NDVI values and flags temporal windows where the decline exceeds a 
predefined threshold, returning the onset date of the potential infestation event.

\subsection{Time-Series Representation and Smoothing}
In this study, we used a 


\subsection{Proposed anomaly detection model}

\subsubsection{Problem Formulation}

\subsubsection{Baseline Time-Series Modeling}

\subsubsection{Anomaly Scoring}

\subsection{Evaluation Metrics}

\clearpage

\section{Results and Discussion}

\subsection{Proof of Concept: Plastic deterioration detection in Mossoró with Sentinel-2} 

\subsection{Proof of Concept: FAW infestation detection in Southern India with Sentinel-2}

\subsubsection{Spectral Signatures of Pest Infestation}
Figure \ref{fig:india_spectral_severity_classification} illustrates the spectral 
signatures associated with varying severity levels of FAW infestation in Southern India.

\begin{figure}[H]
    \centering
    \includegraphics[width=1\textwidth]{figures/india_spectral_severity_classification.png}
    \caption{Spectral signatures of maize crops under different severity levels of Fall Armyworm infestation in Southern India.}
    \label{fig:india_spectral_severity_classification}
\end{figure}

Based on the mean spectral response of the field ilustrated in Figure \ref{fig:india_spectral_severity_classification},
we can interpret the its results based on each spectral band characteristics. Near-Infrared (B8): 
carries information about leaf structure. Damage from pest feeding reduces reflectance in this 
band. Blue (B2) and Red (B4) carry chlorophyll information. Chlorophyll absorbs these wavelengths; 
reduction in absorption indicates decreased photosynthetic activity, often caused by pest 
feeding on leaves. Shortwave Infrared (B10-B12) are sensitive to water content. 
Pest-damaged plants often exhibit water stress, detectable in these bands.

Harvest events cause abrupt transitions from high NDVI (green, healthy) to low NDVI 
(senescence, yellow/brown), which must be distinguished from pest-related declines.
Pest-related


\begin{figure}[H]
    \centering
    \includegraphics[width=1\textwidth]{figures/india_ndvi_time_series_comparison.png}
    \caption{Mean NDVI time series comparison between FAW-infested and healthy maize crops in Southern India.}
    \label{fig:india_ndvi_time_series_comparison}
\end{figure}

\begin{figure}[H]
    \centering
    \includegraphics[width=1\textwidth]{figures/india_ndvi_time_series_maps.png}
    \caption{Mean NDVI time series comparison between FAW-infested and healthy maize crops in Southern India.}
    \label{fig:india_ndvi_time_series_maps}
\end{figure}

\begin{figure}[H]
    \centering
    \includegraphics[width=1\textwidth]{figures/india_ndvi_time_series_maps_healthy.png}
    \caption{Mean NDVI time series comparison between FAW-infested and healthy maize crops in Southern India.}
    \label{fig:india_ndvi_time_series_maps_healthy}

\end{figure}

\subsubsection{Temporal Anomaly Detection}

\clearpage

\section*{References}

\begin{enumerate}
    \item Bilintoh, T.M., et al. (2019). Impact of Armyworm infestations on crops in Ghana: Field validation study.
    
    \item Huete, A., et al. (1997). Overview of the radiometric and biophysical performance of the MODIS vegetation indices. \textit{Remote Sensing of Environment}.
    
    \item Mudereri, B.T., et al. (2022). Application of remote sensing indices for crop health assessment. \textit{Journal of Agricultural Science}.
    
    \item Rouse, J.W., et al. (1974). Monitoring vegetation systems in the Great Plains with ERTS. \textit{Third ERTS Symposium}.
    
    \item Torgbor, B.A., et al. (2022). Remote sensing for pest detection in smallholder agriculture. \textit{Remote Sensing Applications}.
    
    \item Weiss, M., and Baret, F. (2016). S2ToolBox Level 2 products: LAI, FAPAR, FCOVER. \textit{ESA Sentinel-2 Documentation}.
    
    \item Zamanzadeh Darban, Z., et al. (2024). Time series anomaly detection: Methods and applications. \textit{Pattern Recognition}.
    
    \item Zheng, Q., et al. (2018). A review of remote sensing applications in pest and disease detection. \textit{Computers and Electronics in Agriculture}.
\end{enumerate}

\subsection*{Online Resources}
\begin{enumerate}
    \item Time Series Smoothing Methods Tutorial. Kaggle. 
    
    \texttt{kaggle.com/code/furkannakdagg/time-series-smoothing-methods-tutorial}
    
    \item Smoothing Techniques for Time Series Data. Medium. 
    
    \texttt{medium.com/@srv96/smoothing-techniques-for-time-series-data}
    
    \item Introduction to Time Series Forecasting: Smoothing Methods. Medium/Codex.
    
    \item Makerere Fall Armyworm Crop Challenge. Zindi Africa. 
    
    \texttt{zindi.africa/competitions/makerere-fall-armyworm-crop-challenge}
\end{enumerate}

\end{document}