\documentclass[12pt]{article}
\setlength{\parskip}{0.5cm}
\usepackage[margin=1in]{geometry}
\usepackage{setspace}
\usepackage{titling}
\usepackage{lmodern}
\usepackage{ragged2e}
\usepackage{graphicx}
\usepackage{amsmath}
\usepackage{amssymb}


\begin{document}

\begin{titlepage}
    \centering
    \begin{figure}
        \centering
        \begin{minipage}{0.4\textwidth}
            \centering
            \includegraphics[width=0.8\textwidth]{Figures/logo_nipe.png}
        \end{minipage}
        \hfill
        \begin{minipage}{0.3\textwidth}
            \centering
            \includegraphics[width=0.8\textwidth]{Figures/unicamp.png}
        \end{minipage}
    \end{figure}

    {\Large \textbf{Research Internship Abroad (BEPE/FAPESP)}}\\[1cm]

    {\LARGE \textbf{Remote Sensing Anomaly Detection in Time-Series Imagery and Application Development in the Agricultural Context}}\\[1.5cm]

    {\large Process: 2025/22483-0}\\[0.5cm]

    {\large Final Report: December/01/2025 to March/01/2026 (3 months)}\\[3cm]

    \begin{flushright}
    {\large \textbf{Brazil Advisor:} Dr. Rubens Augusto Camargo Lamparelli}\\
    {\large \textbf{UK Supervisor:} Dr. Jefersson Alex dos Santos}\\
    {\large \textbf{UK Co-Supervisor:} Prof. Po Yang}\\
    {\large \textbf{Fellow:} Fernando Kenzo Imami Fugihara}\\
    \end{flushright}

    \vfill

    {\large{Campinas, SP}}\\
    {\large{2025}}
\end{titlepage}

\pagenumbering{gobble} % No page numbers on title page

\newpage
\pagenumbering{arabic} % Start page numbering

\section*{Abstract}
The use of plastic in agriculture has increased significantly in recent years, 
bringing both benefits and environmental challenges. While agricultural plastics 
improve crop yields and resource efficiency, they also lead to the accumulation of 
plastic waste in rural areas. Remote Sensing (RS) data, combined with advanced 
machine learning and computer vision techniques, provide an effective means to 
monitor plasticulture dynamics. Therefore, this research internship aims to explore 
RS anomaly detection (RSAD) techniques in time-series imagery and apply them to 
agricultural monitoring, particularly in detecting subtle cases that involve spectral 
changes such as material deterioration and pest-related disturbances Therefore, I'll 
join researchers at the University of Sheffield to learn RSAD techniques in the agriculture 
context and then explore whether they can be applied in plasticulture. In parallel, I'll 
collaborate with researchers at the University of Sheffield and contribute to the PEZEGO 
pest-management app. The internship will provide hands-on experience in scalable application 
design, app optimization, and model integration, which can enhance our ongoing application, 
GeoHuman. The University of Sheffield was chosen due to its internationally recognized 
expertise in application development, machine learning, computer vision, and remote sensing. 
This experience will strengthen our project in Brazil by improving the accuracy and 
scalability of agricultural monitoring systems. Upon my return, I will disseminate the 
knowledge gained through workshops and collaborative activities with my research group 
at UNICAMP to foster innovation and capacity building in remote sensing applications. 


This report summarizes the activities and outcomes of the research internship 
conducted from December 01, 2025 to March 01, 2026. The internship focused on 
remote sensing anomaly detection in time-series imagery and the development 
of applications in the agricultural context. Key achievements include the 
implementation of novel algorithms for anomaly detection, analysis of 
time-series data.

\break

\section{Introduction}

\section{Methodology}

\subsection{Short Review of Methods (December, 1st - January, 1st)}

In this project the focus is on detecting anomalies on the agricultural fields, in particular,
the use of pest-net in the plasticulture fields, and the pest-attacks in the crops. Given this
context, methods of anomaly detection in time-series imagery can be used to detected these 
devitations. Therefore, a short review of these RSAD methods was made to selected and therefore
study how to approach this problems.

\subsection{Anomaly Detection Models (December, 1st - March, 1st)}

\subsubsection{Sensors and Study Regions}
We’ll be using the Harmonized Sentinel-2 (S2) Level-2A surface reflectance images, accessed 
via Google Earth Engine (GEE) (Google 2023). The Copernicus Sentinel-2 mission features 
two polar-orbiting satellites (2A and 2B) in a sun-synchronous orbit at an altitude of 
786 km, phased 180° apart, which enables a 5-day revisit at the equator under cloud-free 
conditions, extending to 2–3 days at mid-latitudes. 

The study area is located in Mossoró, Rio Grande do Norte, Brazil, where previous field 
assessments and local expertise provide a solid foundation for this research. The region 
is particularly notable for its extensive use of plasticulture in melon cultivation, 
making it a relevant and representative site for analyzing agricultural land-cover 
dynamics. Additionally, pest-management practices, such as the use of pest nets, are 
commonly employed to protect melon crops. Field observations conducted by the research 
team also confirmed that plastic mulch is often reused across multiple cultivation cycles, 
resulting in areas with visibly deteriorated plastic materials. Figure \ref{fig:study_area_mossoro} illustrates the 
landscape of the study region.


\clearpage 
\begin{figure}
    \centering
    \includegraphics[width=0.9\textwidth]{Figures/study_area_mossoro.png}
    \caption{Mossoró region of study, Brazil, Rio Grande do Norte.}
    \label{fig:study_area_mossoro}
\end{figure}

The second study area is located in Ejura, Ghana, West Africa. The region is 
characterized by a wide diversity of crops, including plantain, maize, yam, 
rice, beans, cassava, groundnuts, and watermelon. This agricultural diversity 
makes Ejura a suitable region for studying and detecting crop anomalies, such 
as pest attacks. In particular, the region is highly relevant due to the study 
conducted by Bilintoh et al. (2019), which investigated the impact of Armyworm 
infestations on crops with field validation. Furthermore, this study area aligns 
with the objectives of the ongoing Brazil–UK–Africa collaboration known as 
“SmartPest-Ghana: Exploring LLM-driven mobile solutions for climate-smart 
pest management in maize farming”, funded by UK Research and Innovation (UKRI). 
This project represents a recently established short-term partnership between 
Brazilian and UK research groups and provides a strong contextual and scientific 
foundation for the selection of Ejura as a study area. 
Figure \ref{fig:study_area_ejura}, presents the landscape of the Ejura study region.

\clearpage 
\begin{figure}
    \centering
    \includegraphics[width=0.7\textwidth]{Figures/study_area_ejura.png}
    \caption{Ejura region of study, Ghana, West Africa.}
    \label{fig:study_area_ejura}
\end{figure}

\subsubsection{Time-Series}

In this project, we'll be using time-series imagery from Sentinel-2 satallite. Where the atomic
unit is a single pixel observed through time. In time-series, we have the univariate time-series 
(UTS) and the multivariate time-series (MTS).

A \textbf{UTS} is a series of data that is based on a single variable (such as NDVI) that changes over time,
Therefore, the UTS ${X}$ with ${t}$ timestamps (\textbf{for a single pixel}) can be represented as an ordered sequence of 
data points in the following way (Zamanzadeh Darban et al., 2024):

\[
X
= (x_1, x_2,...,x_t)
\]

Where $x_i$ represents the feature (NDVI) at timestamp $i \in T$ and $T = \{1, 2, \ldots, t\}$.

\[
X^{(i)} = \left( x^{(i)}_1, x^{(i)}_2, \ldots, x^{(i)}_t \right)
\]

where $i$ denotes the pixel, $t$ is the number of timestamps, and $x^{(i)}_t$ represents 
the NDVI value (feature vector) observed at timestamp $t$ for pixel $i$.

A \textbf{MTS} represents multiple variables that are dependent on time, each of which is influenced 
by both past values (\emph{temporal} dependency) and other variables (dimensions) based on their correlation.
The correlations between different variables are referred to as spatial dependencies in the literature.

Consider an MTS represented as a sequence of vectors over time, each vector at time $i$, $X_i$, 
consisting of $d$ dimensions:
\begin{equation}
X = (X_1, X_2, \ldots, X_t)
  = \bigl(
    (x_1^1, x_1^2, \ldots, x_1^d),
    (x_2^1, x_2^2, \ldots, x_2^d),
    \ldots,
    (x_t^1, x_t^2, \ldots, x_t^d)
    \bigr)
\end{equation}

Where $X_i = (x_i^1, x_i^2, \ldots, x_i^d)$ represents a data vector at time $i$, with each $x_i^j$ 
indicating the observation at time $i$ for the $j$th dimension, and $j = 1, 2, \ldots, d$, where $d$ 
is the total number of dimensions.

\subsubsection{Interpretation of the Time-Series Representation}

In the adopted formulation, the symbols $x_1, x_2, \ldots, x_T$ represent the temporal evolution of a variable observed at a fixed spatial location. Specifically, for a given pixel $i$, the sequence

\[
X^{(i)} = \left( x^{(i)}_1, x^{(i)}_2, \ldots, x^{(i)}_t \right)
\]

corresponds to the NDVI values measured at that pixel across $t$ different timestamps.

Each element $x^{(i)}_t$ denotes the NDVI value of pixel $i$ at timestamp $t$. Therefore, a 
single pixel gives rise to a univariate time series describing the temporal behavior of 
vegetation at that specific spatial location.

The index $i$ is used to identify individual pixels. For an image of spatial dimensions 
$H \times W$, the total number of pixels is given by

\[
N = H \times W.
\]

In our analysis the Sentinel-2 imagery has a spatial resolution of $256 \times 256$ pixels, then

\[
N = 256 \times 256 = 65{,}536
\]

distinct pixels are present. Consequently, the dataset consists of $65{,}536$ independent univariate time series, one for each pixel, each of length $T$.

Formally, the complete dataset can be expressed as the collection

\[
\left\{ X^{(1)}, X^{(2)}, \ldots, X^{(N)} \right\},
\quad \text{with} \quad
X^{(i)} \in \mathbb{R}^T.
\]

\subsubsection{Density Estimation}

We're going to model a probability density function (pdf), which gives the probability of
any given n feature vector. Assuming that $\vec{x}^{(i)}$ are independent from each other,
we have that

\[
p(\vec{x})
= p(x_1;\mu_1,\sigma_1^2)\, p(x_2;\mu_2,\sigma_2^2)\, p(x_3;\mu_3,\sigma_3^2)\, \cdots\, p(x_n;\mu_n,\sigma_n^2)
\]

\[
p(\vec{x}) = \prod_{j=1}^{n} p(x_j;\mu_j,\sigma_j^2)
\]

For tractability, we assume conditional independence across timestamps, although 
temporal correlations are known to exist.

\subsubsection{Algorithm}

\begin{enumerate}
\item Choose \( n \) features \( x_i \) that you think might be indicative of anomalous examples.
In our case, the pixel features that could be an indicative of anomalous are the NDVI and EVI, 
spectral indexes.

\item Fit parameters \( \mu_1, \ldots, \mu_n, \sigma_1^2, \ldots, \sigma_n^2 \):
\end{enumerate}

\[
\mu_j = \frac{1}{m} \sum_{i=1}^{m} x_j^{(i)}
\]

\[
\sigma_j^2 = \frac{1}{m} \sum_{i=1}^{m} \left( x_j^{(i)} - \mu_j \right)^2
\]

\[
\vec{\mu} = \frac{1}{m} \sum_{i=1}^{m} \vec{x}^{(i)}
\]

\begin{enumerate}
\setcounter{enumi}{2}
\item Given new example \( \vec{x} \), compute \( p(\vec{x}) \):
\end{enumerate}

\[
p(\vec{x})
= \prod_{j=1}^{n} p(x_j; \mu_j, \sigma_j^2)
= \prod_{j=1}^{n}
\frac{1}{\sqrt{2\pi}\,\sigma_j}
\exp\!\left(
-\frac{(x_j - \mu_j)^2}{2\sigma_j^2}
\right)
\]

\[
\text{Anomaly if } p(\vec{x}) < \varepsilon
\]

\subsubsection{Generating the Map}


\section{Results and Discussion}

\subsection{Spectral Response Behavior}
\clearpage
\begin{figure}
    \centering
    \includegraphics[width=1\textwidth]{Figures/s2_NDVI_EVI.png}
    \caption{Sentinel-2 NDVI and EVI time-series from 2017 to 2021 of crops from Ejura region of study.}
    \label{fig:s2_spectral_response}
\end{figure}

\clearpage
\begin{figure}
    \centering
    \includegraphics[width=1\textwidth]{Figures/l8_NDVI_EVI.png}
    \caption{Landsat 8 NDVI and EVI time-series from 2017 to 2021 of crops from Ejura region of study.}
    \label{fig:l8_spectral_response}
\end{figure}



\section{Conclusion}

\section*{References}

\end{document}