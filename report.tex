\documentclass[12pt]{article}
\setlength{\parskip}{0.5cm}
\usepackage[margin=1in]{geometry}
\usepackage{setspace}
\usepackage{titling}
\usepackage{lmodern}
\usepackage{ragged2e}
\usepackage{graphicx}
\usepackage{amsmath}
\usepackage{amssymb}


\begin{document}

\begin{titlepage}
    \centering
    \begin{figure}
        \centering
        \begin{minipage}{0.4\textwidth}
            \centering
            \includegraphics[width=0.8\textwidth]{Figures/logo_nipe.png}
        \end{minipage}
        \hfill
        \begin{minipage}{0.3\textwidth}
            \centering
            \includegraphics[width=0.8\textwidth]{Figures/unicamp.png}
        \end{minipage}
    \end{figure}

    {\Large \textbf{Research Internship Abroad (BEPE/FAPESP)}}\\[1cm]

    {\LARGE \textbf{Remote Sensing Anomaly Detection in Time-Series Imagery and Application Development in the Agricultural Context}}\\[1.5cm]

    {\large Process: 2025/22483-0}\\[0.5cm]

    {\large Final Report: December/01/2025 to March/01/2026 (3 months)}\\[3cm]

    \begin{flushright}
    {\large \textbf{Brazil Advisor:} Dr. Rubens Augusto Camargo Lamparelli}\\
    {\large \textbf{UK Supervisor:} Dr. Jefersson Alex dos Santos}\\
    {\large \textbf{UK Co-Supervisor:} Prof. Po Yang}\\
    {\large \textbf{Fellow:} Fernando Kenzo Imami Fugihara}\\
    \end{flushright}

    \vfill

    {\large{Campinas, SP}}\\
    {\large{2025}}
\end{titlepage}

\pagenumbering{gobble} % No page numbers on title page

\newpage
\pagenumbering{arabic} % Start page numbering

\section*{Abstract}
The use of plastic in agriculture has increased significantly in recent years, 
bringing both benefits and environmental challenges. While agricultural plastics 
improve crop yields and resource efficiency, they also lead to the accumulation of 
plastic waste in rural areas. Remote Sensing (RS) data, combined with advanced 
machine learning and computer vision techniques, provide an effective means to 
monitor plasticulture dynamics. Therefore, this research internship aims to explore 
RS anomaly detection (RSAD) techniques in time-series imagery and apply them to 
agricultural monitoring, particularly in detecting subtle cases that involve spectral 
changes such as material deterioration and pest-related disturbances Therefore, I'll 
join researchers at the University of Sheffield to learn RSAD techniques in the agriculture 
context and then explore whether they can be applied in plasticulture. In parallel, I'll 
collaborate with researchers at the University of Sheffield and contribute to the PEZEGO 
pest-management app. The internship will provide hands-on experience in scalable application 
design, app optimization, and model integration, which can enhance our ongoing application, 
GeoHuman. The University of Sheffield was chosen due to its internationally recognized 
expertise in application development, machine learning, computer vision, and remote sensing. 
This experience will strengthen our project in Brazil by improving the accuracy and 
scalability of agricultural monitoring systems. Upon my return, I will disseminate the 
knowledge gained through workshops and collaborative activities with my research group 
at UNICAMP to foster innovation and capacity building in remote sensing applications. 


This report summarizes the activities and outcomes of the research internship 
conducted from December 01, 2025 to March 01, 2026. The internship focused on 
remote sensing anomaly detection in time-series imagery and the development 
of applications in the agricultural context. Key achievements include the 
implementation of novel algorithms for anomaly detection, analysis of 
time-series data.

\break

\section{Short Review of Methods (December, 1st - January, 1st)}

In this project the focus is on detecting anomalies on the agricultural fields, in particular,
the use of pest-net in the plasticulture fields, and the pest-attacks in the crops. Given this
context, methods of anomaly detection in time-series imagery can be used to detected these 
devitations. Therefore, a short review of these RSAD methods was made to selected and therefore
study how to approach this problems.

\section{Sensors and Study Regions}
We’ll be using the Harmonized Sentinel-2 (S2) Level-2A surface reflectance images, accessed 
via Google Earth Engine (GEE) (Google 2023). The Copernicus Sentinel-2 mission features 
two polar-orbiting satellites (2A and 2B) in a sun-synchronous orbit at an altitude of 
786 km, phased 180° apart, which enables a 5-day revisit at the equator under cloud-free 
conditions, extending to 2–3 days at mid-latitudes. 

The study area is located in Mossoró, Rio Grande do Norte, Brazil, where previous field 
assessments and local expertise provide a solid foundation for this research. The region 
is particularly notable for its extensive use of plasticulture in melon cultivation, 
making it a relevant and representative site for analyzing agricultural land-cover 
dynamics. Additionally, pest-management practices, such as the use of pest nets, are 
commonly employed to protect melon crops. Field observations conducted by the research 
team also confirmed that plastic mulch is often reused across multiple cultivation cycles, 
resulting in areas with visibly deteriorated plastic materials. Figure \ref{fig:study_area_mossoro} illustrates the 
landscape of the study region.


\clearpage 
\begin{figure}
    \centering
    \includegraphics[width=0.9\textwidth]{Figures/study_area_mossoro.png}
    \caption{Mossoró region of study, Brazil, Rio Grande do Norte.}
    \label{fig:study_area_mossoro}
\end{figure}

The second study area is located in Ejura, Ghana, West Africa. The region is 
characterized by a wide diversity of crops, including plantain, maize, yam, 
rice, beans, cassava, groundnuts, and watermelon. This agricultural diversity 
makes Ejura a suitable region for studying and detecting crop anomalies, such 
as pest attacks. In particular, the region is highly relevant due to the study 
conducted by Bilintoh et al. (2019), which investigated the impact of Armyworm 
infestations on crops with field validation. Furthermore, this study area aligns 
with the objectives of the ongoing Brazil–UK–Africa collaboration known as 
“SmartPest-Ghana: Exploring LLM-driven mobile solutions for climate-smart 
pest management in maize farming”, funded by UK Research and Innovation (UKRI). 
This project represents a recently established short-term partnership between 
Brazilian and UK research groups and provides a strong contextual and scientific 
foundation for the selection of Ejura as a study area. 
Figure \ref{fig:study_area_ejura}, presents the landscape of the Ejura study region.

\clearpage 
\begin{figure}
    \centering
    \includegraphics[width=0.7\textwidth]{Figures/study_area_ejura.png}
    \caption{Ejura region of study, Ghana, West Africa.}
    \label{fig:study_area_ejura}
\end{figure}

\section{Spectral Vegetation Indexes}
Remotely sensed spectral vegetation indices (SVI) are widely used to assess crop health, 
particularly in detecting impacts of diseases and pest invasions (Zheng et al., 2018). 
Among these indices, the Normalised Difference Vegetation Index (NDVI), Enhanced Vegetation Index 
(EVI), Normalised Difference Water Index (NDWI), and Red-edge Chlorophyll Index (RECI) are 
recognised as significant indicators of crop health and stress 
(Torgbor et al., 2022, Mudereri et al., 2022, Zheng et al., 2018).

\begin{table}[ht]
\centering
\caption{Spectral vegetation indices and corresponding equations using Sentinel-2 bands.}
\label{tab:veg_indices}
\renewcommand{\arraystretch}{1.4}
\small
\begin{tabular}{p{2.8cm} p{4.8cm} p{3cm} p{2.2cm}}
\textbf{Indice} & \textbf{Equation} & \textbf{S2 bands} & \textbf{Reference} \\

NDVI & 
$\text{NDVI} = \frac{NIR - Red}{NIR + Red}$ & 
$\frac{B8 - B4}{B8 + B4}$ & 
Rouse et al., 1974 \\

EVI & 
$\text{EVI} = 2.5 \frac{NIR - R}{NIR + 6R - 7.5B + 1}$ & 
$\frac{B8 - B4}{B8 + 6B4 - 7.5B2 + 1}$ & 
Huete et al., 1997 \\

LAI &
$\text{LAI} = 3.618 \times \text{NDVI} - 0.118$ & 
$3.618 \times \frac{B8 - B4}{B8 + B4} - 0.118$ & 
Weiss and Baret, 2016 \\
\end{tabular}
\end{table}


\section{Armyworm Life Cycle}

\subsection{Clustering Time Window}
A pratical use of this research is to alert to the farmers about a possible pest attack in their
crops. In order to this, we need a way to detect the pest infestation as quick as possible and create 
an alert


\section{Spectral reflectance of Fall Armyworm infestation}


\section{Time-Series Definition}

In this project, we'll be using time-series imagery from Sentinel-2 satallite. Where the atomic
unit is a single pixel observed through time. In time-series, we have the univariate time-series 
(UTS) and the multivariate time-series (MTS).

A \textbf{UTS} is a series of data that is based on a single variable (such as NDVI) that changes over time,
Therefore, the UTS ${X}$ with ${t}$ timestamps (\textbf{for a single pixel}) can be represented as an ordered sequence of 
data points in the following way (Zamanzadeh Darban et al., 2024):

\[
X
= (x_1, x_2,...,x_t)
\]

Where $x_i$ represents the feature (NDVI) at timestamp $i \in T$ and $T = \{1, 2, \ldots, t\}$.

A \textbf{MTS} represents multiple variables that are dependent on time, each of which is influenced 
by both past values (\emph{temporal} dependency) and other variables (dimensions) based on their correlation.
The correlations between different variables are referred to as spatial dependencies in the literature.

Consider an MTS represented as a sequence of vectors over time, each vector at time $i$, $X_i$, 
consisting of $d$ dimensions:
\begin{equation}
X = (X_1, X_2, \ldots, X_t)
  = \bigl(
    (x_1^1, x_1^2, \ldots, x_1^d),
    (x_2^1, x_2^2, \ldots, x_2^d),
    \ldots,
    (x_t^1, x_t^2, \ldots, x_t^d)
    \bigr)
\end{equation}

Where $X_i = (x_i^1, x_i^2, \ldots, x_i^d)$ represents a data vector at time $i$, with each $x_i^j$ 
indicating the observation at time $i$ for the $j$th dimension, and $j = 1, 2, \ldots, d$, where $d$ 
is the total number of dimensions.

\section{Time Series Basic Structure}

\subsection{Level}
It is the average value of the time-series. It can be thought of as the mean of the series.

\subsection{Stationary}
A stationary time-series is a sequence of data where statistical properties of the series do 
not change over time. In other words, if the mean, variance and covariance of a time series 
remain constant over time, the series is said to be stationary.

\subsection{Trend}
It is the structure of the long-term increase or decrease of a time series. If there is a trend,
it is very unlikely that the series will be stationary because the statistics of the periods 
(mean, standard deviation, etc.) will change in an increasing or decreasing trend.

\subsection{Seasonality}
Seasonality is when a time series repeats a certain behavior at certain intervals. 

\subsection{Cycle}
It contains repetitive patterns similar to seasonality and these two issues can be confused with
each other. Seasonality can be mapped to specific time periods. It overlaps with structures such 
as day, week, year, season. For example, markets do more business on weekends or a product gets 
more attention in winter, etc. The cyclicity takes place in a longer time, in a more uncertain 
structure, in a way that does not overlap with structures such as day, week, year, season. It 
occurs mostly for structural reasons, with cyclical changes. For example, it is shaped by the 
speeches of some people from the business world and the speeches of politicians. Although this i
s not completely seasonal, it occurs in a certain period, but the period in which it will occur 
is not clear. 

\section{Time-series Smoothing}
With Jefersson I've got contact with different time-series smoothing methods. Those are very
useful to extract more information of the series and avoid some common mistakes when analysing
the data smoothing techniques are kinds of data preprocessing techniques to remove noise from a 
data set. This allows important patterns to stand out. 

For example, let's say we have two pixels  with different time series responses, where 
one series in like a sine and the other like a cosine. If we decide to utilize in this series a 
clustering algorithm which is based on the distance between these values, the difference will 
be high, but in reality the distance is not that, but because the data is noisy it creates this
misconception. Therefore, is really important to use time-series smoothing when utilizing time-series.

There are several time-series smoothing techniques such as Moving Average Smoothing, Weighted Average Smoothing, 
Single Exponential Smoothing (SES), Double Exponential Smoothing (DES), Triple Exponential Smoothing (TES) among others.
Their use depends on the time-series structure, as presented in Table \ref{tab:t1}.

\begin{table}[h]
\centering
\caption{Smoothing Algorithm usecases in different time-series structures}
\label{tab:t1}
\renewcommand{\arraystretch}{1.3}
\begin{tabular}{|l|c|c|c|c|}
\hline
\textbf{Algorithm} & \textbf{Level} & \textbf{Trend} & \textbf{Seasonality} & \textbf{Tuning Parameters} \\
\hline
Single HWES  & Yes & No  & No  & $\alpha$ \\
Double HWES  & Yes & Yes & No  & $\alpha, \beta$ \\
Triple HWES  & Yes & Yes & Yes & $\alpha, \beta, \gamma$ \\
\hline
\end{tabular}
\end{table}

The variables used in the smoothing methods are presented in Table \ref{tab:t2}.

\begin{table}[h]
\centering
\caption{Variables utilized in the smoothing models}
\label{tab:t2}
\renewcommand{\arraystretch}{1.4}
\begin{tabular}{|c|l|}
\hline
\textbf{Symbol} & \textbf{Description} \\
\hline
$X$ & Observation \\
$S$ & Smoothed observation \\
$B$ & Trend factor \\
$C$ & Seasonal index \\
$F$ & The forecast at $m$ periods ahead \\
$\alpha$ & Data smoothing factor, $\alpha \in (0,1)$ \\
$\beta$ & Trend smoothing factor, $\beta \in (0,1)$ \\
$\gamma$ & Seasonal change smoothing factor, $\gamma \in (0,1)$ \\
$\phi$ & Damped smoothing factor, $\phi \in (0,1)$ \\
$t$ & The index that denotes a time period \\
\hline
\end{tabular}
\end{table}


\subsection{Moving Average and Weighted Average Smoothing}
The future value of a time series is the average of its k previous values. Moving average 
trading is generally used in practice not for forecasting, but for capturing/observing a trend. 
However, while deriving features within the scope of machine learning, we are still generating 
features based on moving averages. Similar to a moving average. Weighted average carries the idea of giving more weight to later observations.

Although this methods are very useful, they cannot be used in our context because they do not
consider trend and seasonality, which are present in our satelite time-series.

\subsection{Single Exponential Smoothing (SES)}
SES = Level (Unsuccessful if Trend and Seasonality).

It is successful only in stationary series. There should be no trends and seasonality. 
It can model the level(Level can be thought of as the average of the series.). 
The effects of the past are weighted on the assumption that "the future is more related 
to the recent past". SES is suitable for UTS without trend and seasonality, 
it is successful in stationary series.

\[
S_0 = X_0
\]

\[
S_t = \alpha X_t + (1 - \alpha) S_{t-1}
\]

\[
\{\, t > 0,\; 0 < \alpha < 1 \,\}
\]


\subsection{Double Exponential Smoothing (DES)}
DES = Level (SES) + Trend

The basic approach is the same. In addition to SES, the trend is also taken into account. 
It is suitable for univariate time series with and without seasonality.

 \[
\begin{aligned}
S_0 &= X_0 \\
B_0 &= X_1 - X_0 \\
\\
S_t &= \alpha X_t + (1 - \alpha)(S_{t-1} + B_{t-1}) \\
B_t &= \beta (S_t - S_{t-1}) + (1 - \beta) B_{t-1} \\
\\
\alpha, \beta &\in (0,1)
\end{aligned}
\]


\subsection{Triple Exponential Smoothing (TES)}
TES = SES + DES + Seasonality

It is the most advanced smoothing method. This method makes predictions by evaluating 
the effects of level, trend and seasonality dynamically. It can be used in UTS with 
trend and/or seasonality.

\[
\begin{aligned}
S_0,\, F_0 &= X_0 \\
\\
B_0 &= \frac{\sum_{i=0}^{L-1} \left( X_{L+i} - X_i \right)}{L^2} \\[10pt]
\\
S_t &= \alpha \left( X_t - C_{t \bmod L} \right) + (1 - \alpha)\left( S_{t-1} + \phi B_{t-1} \right) \\[6pt]
\\
B_t &= \beta \left( S_t - S_{t-1} \right) + (1 - \beta)\phi B_{t-1} \\[6pt]
\\
C_{t \bmod L} &= \gamma \left( X_t - S_t \right) + (1 - \gamma) C_{t \bmod L} \\[10pt]
\\
F_{t+m} &= S_t + B_t \sum_{i=1}^{m} \phi^{i} + C_{t \bmod L} \\[10pt]
\\
\alpha, \beta, \gamma &\in (0,1)
\end{aligned}
\]

\section{Anomaly Detection Models}

\subsection{Density Estimation}

We're going to model a probability density function (pdf), which gives the probability of
any given n feature vector. Assuming that $\vec{x}^{(i)}$ are independent from each other,
we have that

\[
p(\vec{x})
= p(x_1;\mu_1,\sigma_1^2)\, p(x_2;\mu_2,\sigma_2^2)\, p(x_3;\mu_3,\sigma_3^2)\, \cdots\, p(x_n;\mu_n,\sigma_n^2)
\]

\[
p(\vec{x}) = \prod_{j=1}^{n} p(x_j;\mu_j,\sigma_j^2)
\]

For tractability, we assume conditional independence across timestamps, although 
temporal correlations are known to exist.

\subsection{Algorithm}

\begin{enumerate}
\item Choose \( n \) features \( x_i \) that you think might be indicative of anomalous examples.
In our case, the pixel features that could be an indicative of anomalous are the NDVI and EVI, 
spectral indexes.

\item Fit parameters \( \mu_1, \ldots, \mu_n, \sigma_1^2, \ldots, \sigma_n^2 \):
\end{enumerate}

\[
\mu_j = \frac{1}{m} \sum_{i=1}^{m} x_j^{(i)}
\]

\[
\sigma_j^2 = \frac{1}{m} \sum_{i=1}^{m} \left( x_j^{(i)} - \mu_j \right)^2
\]

\[
\vec{\mu} = \frac{1}{m} \sum_{i=1}^{m} \vec{x}^{(i)}
\]

\begin{enumerate}
\setcounter{enumi}{2}
\item Given new example \( \vec{x} \), compute \( p(\vec{x}) \):
\end{enumerate}

\[
p(\vec{x})
= \prod_{j=1}^{n} p(x_j; \mu_j, \sigma_j^2)
= \prod_{j=1}^{n}
\frac{1}{\sqrt{2\pi}\,\sigma_j}
\exp\!\left(
-\frac{(x_j - \mu_j)^2}{2\sigma_j^2}
\right)
\]

\[
\text{Anomaly if } p(\vec{x}) < \varepsilon
\]

\subsection{Algorithm 2}

\begin{enumerate}
\item Choose \( n \) features \( x_i \) that you think might be indicative of anomalous examples.
In our case, the pixel features that could be an indicative of anomalous are the NDVI and EVI, 
spectral indexes.

\item Group pixels exhibiting similar temporal behavior using an clustering.
Let each pixel $\vec{x}^{(i)}$ be assigned to a cluster $c \in \{1,\ldots,K\}$.

\item For each cluster $c$, estimate the parameters of a Gaussian distribution independently 
for each feature:
\end{enumerate}

\[
\mu_j^{(c)} = \frac{1}{m_c} \sum_{i=1}^{m_c} x_j^{(i)}, \quad
\sigma_j^{2(c)} = \frac{1}{m_c} \sum_{i=1}^{m_c} \left( x_j^{(i)} - \mu_j^{(c)} \right)^2
\]

\begin{enumerate}
\setcounter{enumi}{3}
\item Given a new example $\vec{x}$ assigned to cluster $c$, compute its likelihood:
\end{enumerate}

\[
p(\vec{x} \mid c)
= \prod_{j=1}^{n} p(x_j; \mu_j^{(c)}, \sigma_j^{2(c)})
= \prod_{j=1}^{n}
\frac{1}{\sqrt{2\pi}\,\sigma_j^{(c)}}
\exp\!\left(
-\frac{(x_j - \mu_j^{(c)})^2}{2\sigma_j^{2(c)}}
\right)
\]

\[
\text{Anomaly if } p(\vec{x} \mid c) < \varepsilon_c
\]

B8 - carrega a informação da estrutura da folha.

AZUL E VERMELHO - carrega a informação da clorofila. clorofila absoverve esses dois comprimentos de onda.
Uma queda nesses nesses dois comprimentos de onda indica que a planta não está realizando fotossintese. Indicando que 
a lagarta está comendo as folhas da planta.

B10 - B12 - água, plantas com pragas podem apresentar um defict de água.

Na colheita, as plantas vao do verde para senesncia(morte) - amarelo, grão. Vai de um NDVI ALTO para um NDVI BAIXO (Bruscamente).

\section*{References}

\begin{enumerate}
    \item https://www.kaggle.com/code/furkannakdagg/time-series-smoothing-methods-tutorial
    \item https://medium.com/@srv96/smoothing-techniques-for-time-series-data-91cccfd008a2
    \item https://medium.com/codex/introduction-to-time-series-forecasting-smoothing-methods-9a904c00d0fd
    \item https://www.sciencedirect.com/science/article/pii/S1569843225001633\#b0330
    \item https://www.tandfonline.com/doi/full/10.1080/10106049.2020.1869330\#d1e287
    \item https://zindi.africa/competitions/makerere-fall-armyworm-crop-challenge/data
    \item https://link.springer.com/article/10.1007/s11119-021-09845-4
\end{enumerate}



\end{document}